\subsection{Priorité}
Une priorité a été ajoutée aux threads. Pour cela, on a ajouté à la fifo une fonction \textbf{addthreadprio()} qui va mettre le thread à la bonne position dans la liste en respectant l'ordre des priorités. Si la priorité du thread 1 est supérieure à celle du thread 2, ce dernier sera placé après le premier.\\
La bibliothèque gère alors les priorités. La fonction \textbf{thread\_create\_prio} permet de créer un thread avec une priorité passée en paramètre. Le thread avec la priorité la plus élevée sera exécuter le premier.\\




\subsection{Préemption}
La préemption consiste en la capacité du programme d'exécuter ou arrêter une tâche. Dans cette partie nous allons décrire notre implémentation de cette fonctionnalité.

\subsection{Mutex}
Une structure de mutex a été implémentée pour notre bibliothèque. C'est une structure \textbf{thread\_spin\_lock} contenant un \textbf{unsigned int} volatile.\\
Pour pouvoir l'utiliser, il faut pouvoir l'initialiser, le verrouiller, le déverrouiller et de le libérer de la mémoire.\\

Pour ce faire, le langage assembleur a été utilisé grâce à \textbf{\_\_asm\_\_} suivi du mot-clè \textbf{\_\_volatile\_\_}. Ce dernier assure que la déclaration \textbf{asm} n'est pas enregistrée avec les autres accès volatils. Cette implémentation en langage assembleur permet d'avoir des opérations atomiques.\\

L'initialisation du \textbf{lock} consiste à lui affecter la valeur 0. Le verrouiller et le déverrouiller représentent des fonctions plus intéressantes.//
Par exemple, pour déverrouiller le \textbf{lock}, On va déplacer le bit de la valeur 1 vers la zone mémoire occupée par le verrou. Cela se fait grâce à l'instruction:\\
\begin{verbatim}
__asm__ __volatile__("movb $1,%0"
			: "=m" (lock->lock) 
			:
			: "memory")
\end{verbatim} 






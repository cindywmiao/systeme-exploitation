La fonction lib\_init initialise la bibliothèque de threads en créant la file de threads en attente et en initialisant le thread courant avec thread\_init. Cette fonction sera appelé dans le premier appel de l'une des autres fonctions.%%..
\\

thread\_init retourne un thread initialisé. Dans cette fonction, on alloue dynamiquement de la mémoire et on initialise les différents champs de la structure thread\_t. On alloue dynamiquement de la mémoire pour la pile du thread et on indique à valgrind où se trouve cette pile associée au thread. Le statut du thread est mis à READY, et son contexte est initialisé avec getcontext (on récupère le contexte courant).\\

thread\_create crée un thread. Ceci en appelant les fonction thread\_init et makecontext. Cette dernière utilise une fonction intermédiaire qui prend en argument le pointeur vers la fonction à exécuter par le thread et un pointeur vers les arguments qu'elle prend. \\
exec, la fonction intermédiaire utilisé dans le makecontext dans thread\_create retourne simplement \\thread\_exit(func(funcarg)). Cette fonction a été utilisé pour contourné le problème du passage d'argument que nous avons en parce que makecontext ne passe que des entiers en argument pour la fonction. Nous avons alors passé des pointeurs qui sont pris comme des entiers pour la fonction makecontext mais qui permettent d'accéder correctement à la fonction et aux arguments au niveau de cette fonction intermédiaire. %% bien exprimé?
\\

thread\_join passe la main avec thread\_yield si le statut du thread passé en argument est READY ou attend la fin de son exécution s'il est actif (son statut est RUNNING). Si le champ retval du thread est nul, il le modifie en lui affectant la valeur passé en argument.\\

thread\_yield

thread\_exit



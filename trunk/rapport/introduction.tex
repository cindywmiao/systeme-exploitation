L'objectif de ce projet est de mettre en oeuvre, sur un cas pratique, les notions et les acquis vus dans le module de système d'exploitation.

Ce projet consiste, dans un premier temps, à  réaliser une bibliothèque de gestion de threads en espace utilisateur à  l'aide du langage C. On devra donc tout d'abord définir une interface de threads permettant de créer, détruire, passer la main à un thread particulier. Cette bibliothèqe de threads doit satisfaire les critères suivants :

   - Les coûts d'ordonnancement sont beaucoup plus faibles

   - Les politiques d'ordonnancement sont configurables

   - Pouvoir expérimenter réellement le changement de contexte


Aussi, nous veillerons à rester relativement proche de la bibliothèque pthread.h afin de pouvoir facilement comparer les deux implémentations avec des programmes similaires.

Dans ce rapport, nous allons présenter le travail que nous avons réalisé à savoir le choix des structures mises en place et l'implémentation de l'interface de gestion de threads et enfin nous présenterons les objectifs visés au futur.

La fonction \textbf{lib\_init} initialise la bibliothèque de threads en créant la file de threads en attente et en initialisant le thread courant avec thread\_init. Cette fonction sera executée dans le premier appel de l'une des autres fonctions.%%..
\\

\textbf{thread\_init} retourne un thread initialisé. Dans cette fonction, on alloue dynamiquement de la mémoire et on initialise les différents champs de la structure thread\_t. On alloue dynamiquement de la mémoire pour la pile du thread et on indique à valgrind où se trouve cette pile associée au thread. Le statut du thread est mis à READY, et son contexte est initialisé avec getcontext (on récupère le contexte courant).\\

\textbf{thread\_create} crée un thread. Ceci en appelant les fonctions thread\_init et makecontext qui utilise une fonction intermédiaire qui prend en argument le pointeur vers la fonction à exécuter par le thread et un pointeur vers les arguments qu'elle prend. Le thread courant est modifié aprés avoir été ajouté à la file waitq. Enfin il est activé avec swapcontext. \\
exec, la fonction intermédiaire utilisée dans le makecontext dans thread\_create retourne simplement \\thread\_exit(func(funcarg)). Cette fonction a été utilisée pour contourner le problème du passage d'argument que nous avons eu parce que makecontext ne passe que des entiers en argument pour la fonction. Nous avons alors passé des pointeurs qui sont pris comme des entiers pour la fonction makecontext mais qui permettent d'accéder correctement à la fonction et aux arguments au niveau de cette fonction intermédiaire. %% bien exprimé?
\\

\textbf{thread\_join} passe la main avec thread\_yield si le statut du thread passé en argument est différent de TERMINATED. Si le champ retval du thread est nul, il le modifie en lui affectant la valeur passée en argument.\\

\textbf{thread\_yield} passe la main au thread suivant. Si la file est non vide, le thread courant est enfilé, et la tête de la file est alors défilée pour être le nouveau thread courant.\\

\textbf{thread\_exit} affecte la valeur pointée par son argument à la valeur de retval, change le statut du thread courant à TERMINATED, et passe la main avec thread\_yield.\\ 



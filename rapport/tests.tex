Des programmes de tests ont été appliqués à notre bibliothèque.\\
Pour l'instant, la majorité des tests passe. Deux tests posent encore problème: \textbf{switch-many-join} et \textbf{join-main}.\\

Quelques tests ne marchent plus essayé avec un argument trop élevé, ceci est en général dû à l'allocation de mémoire avec \textbf{malloc} dans le tas. Ceci dépend bien sûr des capacités de la machine sur laquelle le test a eu lieu.\\
Sur les machines de l'école:\\
\begin{itemize}

\item le test \textbf{exemple} tourne correctement. \\
\item le test \textbf{join} tourne correctement.\\
\item le test \textbf{main} tourne correctement.\\
\item le test \textbf{switch} tourne correctement.\\
\item le test \textbf{create-many} marche avec un argument pouvant aller jusqu'aux alentours de 5000000, c'est à dire 5000000 threads créés puis détruits. On peut augmenter même plus mais le temps d'exécution devient alors très élevé. \\
\item le test \textbf{create-many-recursif} accepte un argument allant jusqu'à 5000. Au delà de cette valeur, le temps d'exécution et soit trop élevé, soit c'est le caractère récursif de la fonction qui a donc une complexité en espace assez élevée ce qui bloque le déroulement du programm.\\
\item le test \textbf{fibonacci} ne peut être exécuter avec un argument de valeur 23. Au delà de cette valeur, on a une erreur d'allocation: on ne peut plus allouer de la mémoire.\\
\item le test \textbf{switch-many} tourne correctement avec un nombre élevé de threads (4000 threads) et un nombre élevé de yield (4000 yield). Par contre, dès que le nombre de threads et de yield augmente, le temps d'éxécution devient trop long, déjà que pour 4000 threads et 4000 yield, le temps d'exécution est d'environ 10 secondes. \\


\end{itemize}

